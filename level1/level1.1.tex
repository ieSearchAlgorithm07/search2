\subsection{Level1.1: OR問題が解けることの確認}
\subsubsection{学習が収束する回数}
指定されたシード値を用いた際の、学習が終了した回数を表
\ref{table:level1.1}に示す。

\begin{table}[htb]
\begin{center}
\caption{OR問題の学習に要した回数}
\label{table:level1.1}
\begin{tabular}[htb]{r|l} \hline
シード値 & 収束した回数 \\ \hline \hline
1000 & 97 \\ \hline
2000 & 91 \\ \hline
3000 & 112 \\ \hline
4000 & 109 \\ \hline
5000 & 94 \\ \hline
6000 & 100 \\ \hline
7000 & 101 \\ \hline
8000 & 115 \\ \hline
9000 & 114 \\ \hline
10000 & 95 \\ \hline
\end{tabular}
\end{center}
\end{table}

今回、シード値を変えて行った10回の実験の結果を、それぞれtxtファイルに保存し、シェルスクリプトを用いてgnuplotでグラフを作成した。また、10回のデータの平均を求めるシェルスクリプトを作成し、gnuplotでグラフを作成した。横軸を学習回数、縦軸を誤差としたときの学習結果のグラフを
図\ref{graph:level1.1.1}と図\ref{graph:level1.1.2}に示す。

\begin{figure}[h]
\begin{center}
\includegraphics[width=10.0cm]{level1/ex.pdf}
\caption{シード値を変更して得られた10回分の学習結果}
\label{graph:level1.1.1}
\end{center}
\end{figure}


\begin{figure}[h]
\begin{center}
\includegraphics[width=10.0cm]{level1/ex_ave.pdf}
\caption{10回分の学習結果を平均した結果}
\label{graph:level1.1.2}
\end{center}
\end{figure}

\subsubsection{考察}
表\ref{table:level1.1}より、シード値が変わっても、収束するまでの学習回数はおおよそ100回ぐらいになる。全体的に、学習回数が増加していくと誤差は収束していくが、序盤は誤差の変化が大きい。\\
図\ref{graph:level1.1.1}と図\ref{graph:level1.1.2}より、シード値の増減でグラフの大きな変化はなかった。
