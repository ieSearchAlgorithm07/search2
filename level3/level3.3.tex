\subsection{Level3.3: 任意の評価用データを用いた評価}
\subsubsection{アプローチ}

(仮説1)\\
学習時のデータ(教師データ)との違いが少ない程認識率が高く、
逆に教師データとの違いが多い程認識率が低くなるとの仮定の下、
明らかに見た目が違う評価データは用意せず教師データを拡大,位置移動,縮小
のような認識しやすい評価データと,認識しにくい斜めにした評価データを用意した。

\subsubsection{結果}

\begin{lstlisting}[caption=学習時のデータ拡大,label=ラベル]
nn> e
filename? --> data.num2/eva1-4.txt
correct = 0100000000
000111110000
000001000000
000001000000
000001000000
000001000000
000001000000
000001000000
000001000000
000001000000
000001000000
000001000000
000111110000
CHECK filename data.num2/eva1-4.txt
EVA o[0] = 0.03804, correct[0] = 0.1
EVA o[1] = 0.64843, correct[1] = 0.9
EVA o[2] = 0.18650, correct[2] = 0.1
EVA o[3] = 0.03347, correct[3] = 0.1
EVA o[4] = 0.16344, correct[4] = 0.1
EVA o[5] = 0.26652, correct[5] = 0.1
EVA o[6] = 0.06497, correct[6] = 0.1
EVA o[7] = 0.07104, correct[7] = 0.1
EVA o[8] = 0.07718, correct[8] = 0.1
EVA o[9] = 0.02484, correct[9] = 0.1
EVA sum_error = 0.85847
\end{lstlisting}


\begin{lstlisting}[caption=学習時のデータ位置移動,label=ラベル]
nn> e
filename? --> data.num2/eva1-5.txt
correct = 0100000000
000000000000
000000001110
000000000100
000000000100
000000000100
000000000100
000000000100
000000000100
000000000100
000000000100
000000001110
000000000000
CHECK filename data.num2/eva1-5.txt
EVA o[0] = 0.06909, correct[0] = 0.1
EVA o[1] = 0.13674, correct[1] = 0.9
EVA o[2] = 0.08040, correct[2] = 0.1
EVA o[3] = 0.44300, correct[3] = 0.1
EVA o[4] = 0.08931, correct[4] = 0.1
EVA o[5] = 0.30921, correct[5] = 0.1
EVA o[6] = 0.05398, correct[6] = 0.1
EVA o[7] = 0.04762, correct[7] = 0.1
EVA o[8] = 0.12550, correct[8] = 0.1
EVA o[9] = 0.24100, correct[9] = 0.1
EVA sum_error = 1.64158
\end{lstlisting}


\begin{lstlisting}[caption=学習時のデータ斜め,label=ラベル]
filename? --> data.num2/eva1-6.txt    
correct = 0100000000
000000000000
000000100000
000000001000
000000100010
000001000000
010010000000
000100000000
000010000000
000000000000
000000000000
000000000000
000000000000
CHECK filename data.num2/eva1-6.txt
EVA o[0] = 0.13168, correct[0] = 0.1
EVA o[1] = 0.22826, correct[1] = 0.9
EVA o[2] = 0.35467, correct[2] = 0.1
EVA o[3] = 0.14085, correct[3] = 0.1
EVA o[4] = 0.06984, correct[4] = 0.1
EVA o[5] = 0.17521, correct[5] = 0.1
EVA o[6] = 0.00947, correct[6] = 0.1
EVA o[7] = 0.34620, correct[7] = 0.1
EVA o[8] = 0.21842, correct[8] = 0.1
EVA o[9] = 0.15081, correct[9] = 0.1
EVA sum_error = 1.61026
\end{lstlisting}

\begin{lstlisting}[caption=学習時のデータ縮小,label=ラベル]
nn> e
filename? --> data.num2/eva1-7.txt 
correct = 0100000000
000000000000
000000000000
000011100000
000001000000
000001000000
000001000000
000001000000
000001000000
000001000000
000011100000
000000000000
000000000000
CHECK filename data.num2/eva1-7.txt
EVA o[0] = 0.08543, correct[0] = 0.1
EVA o[1] = 0.73866, correct[1] = 0.9
EVA o[2] = 0.08469, correct[2] = 0.1
EVA o[3] = 0.04289, correct[3] = 0.1
EVA o[4] = 0.14736, correct[4] = 0.1
EVA o[5] = 0.14030, correct[5] = 0.1
EVA o[6] = 0.03224, correct[6] = 0.1
EVA o[7] = 0.19359, correct[7] = 0.1
EVA o[8] = 0.07265, correct[8] = 0.1
EVA o[9] = 0.10494, correct[9] = 0.1
EVA sum_error = 0.52962
\end{lstlisting}

\subsubsection{考察}
Iを拡大,縮小,斜め,位置移動をしたものを評価用データとして認識テストを行った結果より,
拡大,縮小はほかのと比べエラーが低めだが,斜めそして予想に反して学習時のデータを平行移動させただけのデータでは
エラーがとても高くIとしては到底認識できていないことが分かった.
これよりこの文字の認識には学習時のデータの形はもちろんだがそれよりも位置が非常に重要になっていると予想できる.