\subsection{Level3.4: 認識率を高める工夫}
\subsubsection{対象とする問題点}
入力されたデータが想定していた入力と比べてサイズが異なったり、位置がずれている等、文字の一部が欠けている以外にも多様な要因によるデータ(情報)の劣化による認識率の低下.
\subsubsection{改善方法の提案}
	\begin{itemize}
	\item 学習時のデータをより多く様々な可能性を考え用意することで認識率を高める.
	\item 逆に認識したいデータとは全く違うデータを学習させ,認識率が低い場合に認識したいデータと判断する方法.
	\end{itemize}
\subsubsection{考察}
一つ目の提案では,例えばIを認識したいとき,斜め,拡大,縮小,位置変更をしたIをそれぞれ
複数学習させることにより,その学習時のデータの組み合わせにより様々な形のデータにも
認識率を高めることができると考えた.
二つ目の提案は,認識率が高いことにこだわらず逆に認識率が低いことに着眼点をおくことにより,逆転の発想でデータの認識率を高めることができると考えた.